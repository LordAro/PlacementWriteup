\documentclass[12pt,a4paper]{article}
\usepackage[T1]{fontenc} % Make sure the font is rendered nicely
\usepackage{enumitem} % list indentation
\usepackage[margin=2cm]{geometry}
\usepackage{booktabs}

\title{Industrial Placement Accreditation Module (IPAM)}
\author{Charles Pigott}

\newcommand{\paragraphnl}[1]{\paragraph{#1}\mbox{}\\}

\begin{document}

\maketitle

\section*{Part 1 --- Student and Placement details}

\noindent\begin{tabular}{ll}
	\multicolumn{2}{l}{\textbf{Student details}}              \\
	\toprule
	\textbf{Student name}  & Charles Pigott                   \\
	\midrule
	\textbf{Degree course} & Computer Science and Mathematics \\
	\bottomrule
\end{tabular}

\vspace{2em}

\noindent\begin{tabular}{ll}
	\multicolumn{2}{l}{\textbf{Placement details}}                         \\
	\toprule
	\textbf{Company}                        & Bytemark Hosting Ltd.        \\
	\midrule
	\textbf{Location}                       & York Business Park, YO26 6BL \\
	\midrule
	\textbf{Start date}                     & 2015/07/06                   \\
	\midrule
	\textbf{Workplace supervisor name}      & Matthew Bloch                \\
	\midrule
	\textbf{Workplace supervisor job title} & Managing Director            \\
	\midrule
	\textbf{Placement tutor}                & Paul Keeler                  \\
	\bottomrule
\end{tabular}

\section*{Part 2 --- Role description and learning plan}
\subsection*{Key objectives}
	Bytemark is an internet hosting company founded in 2002 by Matthew Bloch and
	Peter Taphouse that's based in York. The company provides a range of hosting
	services in Manchester and, as of 2014, their brand new data-centre in York.
	The company currently numbers 28 people and is growing quickly, providing
	all sorts of services, from simple virtual machines to powerful dedicated
	servers.

\subsection*{Key responsibilities}
	As a sysadmin, my job will be to help maintain these systems and to help
	customers with any difficulties they may have. When on rota, I will have to
	speak to the customers directly or be available at all hours for more
	urgent queries. When not doing that, I will be helping with the development
	of the in-house software used to manage and control the servers.

\subsection*{Learning plan}
\subsubsection*{Technical}
	As part of my placement, I hope to learn how a hosting provider operates on
	a day-to-day basis and how problems are dealt with. I'll have a main role
	as a Sysadmin, which will mean learning about how to maintain servers and
	fix any problems that may occur.
	As Bytemark own their own data centre and quite a lot of the software used
	is developed in-house, I have the opportunity to learn about the
	``full stack'' of how a hosting provider works, from the software used to
	manage the virtual machines and physical hardware and the processes used in
	developing it, down to the physical machines and cabling itself.

\subsubsection*{Non-technical}
	On the non-technical side of things, I hope that I can improve how I
	interact with people. As I will be working in an office environment with
	others, I'll have to learn to carry out my superior's instructions and how
	to ask others for help when needed. I'll also regularly be talking to
	customers via phone and email, answering their questions and solving their
	problems.

	I also hope to pick up a better working routine, as I am expected to do
	37.5 hours of work every week, and be in the office 9am to 5pm when it is
	my turn on the rota.

\section*{Part 3 --- Learning Journal}
\paragraphnl{July}
	The very first thing I was asked to do was install a Linux distribution on
	the laptop I'd been given to use during my time at Bytemark. This was a
	good refresher for me, as it had been a while since I last did a fresh
	install of a Linux OS, especially with the additional requirement of having
	to use full disk encryption. Initially, I tried to install Arch Linux on
	the laptop, but after being unsuccessful twice and getting conscious of the
	time it was taking, I opted for installing Debian, which while not as
	`bleeding edge' as Arch, has a simpler installer with a builtin option for
	disk encryption.

	One of the first tasks I was given was to create a system that could send
	newsletters to customers. This was beneficial to me as it would mean
	talking to several different people in the company for access and
	clarifying requirements. This enabled me to get to know the people I would
	be working with and let me know the particular system they manage. In
	particular, provisioning of the system that the newsletter software,
	assigning it a subdomain and getting access to the central database server
	all meant talking to different people. A newsletter system would help the
	company to easily contact all existing customers with any updates; new
	product announcements being what was wanted in this case.

	Another job was organising the migration of VMs running on a particular
	host to another, as the existing host had a hardware fault. These VMs were
	running on an older platform and so were not able to be live migrated
	across like the newer platform. Doing this would mean downtime for the
	customers, so they would have to be contacted in advance. As this migration
	was necessary due to a hardware fault, they would only be given a days
	notice. After getting the list of customer's emails, I wrote a script to
	generate emails with their name and virtual machine name, so it was a bit
	more personal. This gave me further reason to get to know my colleagues and
	was my first time interacting with customers, as if any of them replied
	wanting their VM to be migrated earlier (usually as the specified time was
	inconvenient to them), I'd have to email back confirming that I'd done what
	they'd asked.

\paragraphnl{August}
	This month was my first time on support, where my primary role would be to
	answer the phones and answer emails. Taking phone calls wasn't something I
	was very comfortable with, so this was a good opportunity to properly get
	used to it and learn to think fast enough (or be prepared with the
	necessary information in front of me) so I could answer the customer
	properly. Support is something Bytemark prides itself on, and for many
	customers the excellent support is the main reason they choose Bytemark as
	a hosting provider, so it's important for the company that customer's
	questions are answered quickly, correctly, and in a friendly manner.

\paragraphnl{September}
	In September, it came to be my turn to be on out-of-hours support for the
	week.

\paragraphnl{October}
- ip stuff
\paragraphnl{November}
- review?
\paragraphnl{December}
\paragraphnl{January}
\paragraphnl{February}
\paragraphnl{March}
\paragraphnl{April}
\paragraphnl{May}
\paragraphnl{June}
\paragraphnl{July}


\end{document}
