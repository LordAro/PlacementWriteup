\documentclass[12pt,a4paper]{article}
\usepackage[T1]{fontenc} % Make sure the font is rendered nicely
\usepackage{enumitem} % list indentation
\usepackage[margin=2.5cm]{geometry}
\usepackage{booktabs}

\title{Industrial Placement Accreditation Module (IPAM)}
\author{Charles Pigott}

\newcommand{\paragraphnl}[1]{\paragraph{#1}\mbox{}\\}

\begin{document}

\maketitle

\section*{Part 1 --- Student and Placement details}

\noindent\begin{tabular}{ll}
	\multicolumn{2}{l}{\textbf{Student details}}              \\
	\toprule
	\textbf{Student name}  & Charles Pigott                   \\
	\midrule
	\textbf{Degree course} & Computer Science and Mathematics \\
	\bottomrule
\end{tabular}

\vspace{2em}

\noindent\begin{tabular}{ll}
	\multicolumn{2}{l}{\textbf{Placement details}}                         \\
	\toprule
	\textbf{Company}                        & Bytemark Hosting Ltd.        \\
	\midrule
	\textbf{Location}                       & York Business Park, YO26 6BL \\
	\midrule
	\textbf{Start date}                     & 2015/07/06                   \\
	\midrule
	\textbf{Workplace supervisor name}      & Matthew Bloch                \\
	\midrule
	\textbf{Workplace supervisor job title} & Managing Director            \\
	\midrule
	\textbf{Placement tutor}                & Paul Keeler                  \\
	\bottomrule
\end{tabular}

\section*{Part 2 --- Role description and learning plan}
\subsection*{Key objectives}
	Bytemark is an internet hosting company founded in 2002 by Matthew Bloch and
	Peter Taphouse that's based in York. The company provides a range of hosting
	services in Manchester and, as of 2014, their brand new data-centre in York.
	The company currently numbers 28 people and is growing quickly, providing
	all sorts of services, from simple virtual machines to powerful dedicated
	servers.

\subsection*{Key responsibilities}
	As a sysadmin, my job will be to help maintain these systems and to help
	customers with any difficulties they may have. When on rota, I will have to
	speak to the customers directly or be available at all hours for more
	urgent queries. When not doing that, I will be helping with the development
	of the in-house software used to manage and control the servers.

\subsection*{Learning plan}
\subsubsection*{Technical}
	As part of my placement, I hope to learn how a hosting provider operates on
	a day-to-day basis and how problems are dealt with. I'll have a main role
	as a Sysadmin, which will mean learning about how to maintain servers and
	fix any problems that may occur.
	As Bytemark own their own data centre and quite a lot of the software used
	is developed in-house, I have the opportunity to learn about the
	`full stack' of how a hosting provider works, from the software used to
	manage the virtual machines and physical hardware and the processes used in
	developing it, down to the physical machines and cabling itself.

\subsubsection*{Non-technical}
	On the non-technical side of things, I hope that I can improve how I
	interact with people. As I will be working in an office environment with
	others, I'll have to learn to carry out instructions from others and how to
	ask them for help when needed. I'll also regularly be talking to customers
	via phone and email, answering their questions and solving their problems.

	I also hope to pick up a better working routine, as I am expected to do
	37.5 hours of work every week, and be in the office 9am to 5pm when it is
	my turn on the rota.

\section*{Part 3 --- Learning Journal}
\paragraphnl{July}
	The very first thing I was asked to do was install a Linux distribution on
	the laptop I'd been given to use during my time at Bytemark. This was a
	good refresher for me, as it had been a while since I last did a fresh
	install of a Linux OS, especially with the additional requirement of having
	to use full disk encryption. Initially, I tried to install Arch Linux on
	the laptop, but after being unsuccessful twice and getting conscious of the
	time it was taking, I opted for installing Debian, which while not as
	`bleeding edge' as Arch, has a simpler installer with a builtin option for
	disk encryption. From this I learnt to assess options (which distro to
	choose) first, before just choosing the option I am personally more
	familiar with. If I were to reinstall my laptop in the future, I believe
	that I have since learned enough about Linux and how the relevant systems
	work together that I would be able to get Arch working, however Debian is
	still a good choice, as it is the Linux distro that Bytemark's systems most
	commonly use.

	One of the first tasks I was given was to create a system that could send
	newsletters to customers. This was beneficial to me as it would mean
	talking to (and meeting) several different people in the company for access
	and clarifying requirements. This enabled me to get to know the people I
	would be working with and let me know which particular systems they
	`oversee' (no one really has any set systems as such though). In
	particular, provisioning of the system that the newsletter software would
	be running on, getting it assigned to a domain and getting access to the
	central database server all meant talking to different people. A newsletter
	system would help the company to easily contact all existing customers with
	any updates; new product announcements being what was wanted in this case.
	Bytemark uses 3 main (non-verbal) channels for communication. The primary
	one is an instant messaging program with rooms, for general communication.
	This is also where `urgent' alerts first get brought to your attention.
	Second is an internal WordPress blog where you update everyone else on
	things that have been happening, from what you've spent doing today or
	week, something particularly interesting that just happened, to just asking
	everyone a question about something. Lastly, there's email, including
	several `team' mailing lists (I'm subscribed to `development' and
	`support').

	Another job was organising the migration of VMs running on a particular
	host to another, as the existing host had a hardware fault. These VMs were
	running on an older platform and so were not able to be live migrated
	across like the newer platform. Doing this would mean downtime for the
	customers, so they would have to be contacted in advance. As this migration
	was necessary due to a hardware fault, they would only be given a days
	notice. After getting the list of customer's emails, I wrote a script to
	generate emails with their name and virtual machine name, so it was a bit
	more personal. This gave me further reason to get to know my colleagues and
	was my first time interacting with customers, as if any of them replied
	wanting their VM to be migrated earlier (usually as the specified time was
	inconvenient to them), I'd have to email back confirming that I'd done what
	they'd asked.

\paragraphnl{August}
	This month was my first time on support, where my primary role would be to
	answer the phones and answer emails. Taking phone calls wasn't something I
	was very comfortable with, so this was a good opportunity to properly get
	used to it and learn to think fast enough (or be prepared with the
	necessary information in front of me) so I could answer the customer
	properly. Support is something Bytemark prides itself on, and for many
	customers the excellent support is the main reason they choose Bytemark as
	a hosting provider, so it's important for the company that customer's
	questions are answered quickly, correctly, and in a friendly manner. For
	me, customers were generally very happy to talk to me and gave good
	feedback (there's a form that gets sent when an email ticket is resolved).
	In the cases where I was unable to help a customer immediately over the
	phone, I passed them to a colleague, or asked them to email in, where I
	would be able to either spend the time working out what needed to be done
	or pass it to someone who is more suited to that task.

	I also attended a 2-day Golang conference in London this month. As Bytemark
	is tending to use Go for new projects, this was an excellent opportunity
	for me and my colleagues to learn more about the language and improve the
	quality of the code that is produced, resulting in fewer bugs and problems
	later on. The first day was a tutorial session, teaching the basics of Go.
	I already knew most of this, but it was good to have gone over it again
	in a taught session. I also picked up a couple of patterns I hadn't thought
	of before. The second day was the conference proper, with talks. I most
	enjoyed `Stupid Gopher Tricks', talking about corners of the language that
	might not get thought about too often helping me to think about different
	approaches to programming, and `Code Analysis', about static analysis on
	Go code, a topic that interests me generally.

\paragraphnl{September}
	In September, it came to be my turn to be on out-of-hours support for the
	week. This essentially means that I have to deal with any alerts that are
	raised as `urgent' out of normal office hours (at weekend and 5pm-9am).
	Server issues can happen at all hours of the day and part of the support
	service Bytemark provides is 24/7 support for urgent issues. For me, this
	was good practice at setting a good schedule and getting enough sleep, as I
	was getting woken at all hours by alerts like a server running out of disk
	space and breaking websites or a server running out of memory, falling over
	and needing to be rebooted.

	At the end of the month, I was enrolled on a week long Networking course
	with several of my colleagues. This covered basically the whole range of
	networking systems, from how TCP packets are converted into wire signals
	and back again, to (at a high level) how ISPs use BGP to communicate with
	each other and tell each other where the best routes are to a target
	server, essentially `what makes the Internet work'. The whole course was
	very interesting, if intensive. It reminded me that I will always be
	learning new things and as such will need to know how to learn effectively.
	For Bytemark, if I have a better understanding of how networks work and are
	put together, I can work more effectively as a Sysadmin, being better at
	diagnosing issues such as why one server isn't talking to another. For
	example, if one computer isn't talking to another, it's important to get a
	`traceroute' (the route the packets take) from both directions so you can
	see exactly at which point the packets are getting stopped.

\paragraphnl{October}
	For my first major development project, I was tasked with creating a system
	for managing Bytemark's IP addresses. As a hosting provider, Bytemark has a
	lot of IP addresses allocated to different machines which need keeping
	track of. Currently, there are 2 different stores for these IP addresses
	that are completely separate. This, among other things, makes checking if
	an IP address is already in use by something else more difficult than it
	should be and rather slow (for example in the virtual machine imaging
	process). Centralising a store of IP addresses and where they're routed to
	would simplify this greatly and make certain processes faster and apart
	from anything else simplify working out where IP addresses go.

	For this, I initially began programming my own system, however it was soon
	pointed out to me that there was an existing open source project that
	looked like it could do most of what was wanted of my own program. It's
	built expressly with network engineers in mind (who tend to manage lots of
	IPs like I want to) and playing around with its own demo version, entering
	some testing data confirmed that it covered all the necessary points. This
	taught me to make sure I'm not doing unnecessary work and not `reinventing
	the wheel' as often tends to happen in computer systems. Programmers tend
	to not take the time to understand existing code bases, dismissing them as
	badly made and insisting that it gets rewritten. Often it is more effective
	cost-wise and time-wise to work out the problem areas of an existing code
	base and fix them than it is to entirely rewrite it. Going further and
	setting up an instance of the program on Bytemark's systems myself made me
	more certain that this was worth using for our purposes, so the decision
	was made to start converting the existing databases into the open source
	project's format.

	This month I also had my 3 month review, signifying the end of the
	probationary time on my contract. This was a time for me and my supervisor
	to review how I'd been getting on in the company and was achieving what was
	expected of me and what I expected to be achieving. For the company, this
	ensures all their employees are happy with their work and are working well.
	It's also for looking forward, what is expected to have been achieved by
	the time of the next review (normally in the next year but for me this
	would be at the end of my placement). In this I learned that my supervisor
	thought that I had settled in well and was good at interacting with
	customers.

\paragraphnl{November}
	Around this time, the lead developer at Bytemark got very keen on having a
	proper deployment system, with staging servers and production servers
	being completely separate. Being a hosting company, Bytemark has the
	ability to easily start up new systems for just a single application. To
	use this process properly, a deployment program was made use of, using
	Ansible. This enables you to completely wipe an old system, reimage it and
	then deploy your program with a single command. Of course, a staging server
	needs different information (which databases connect to and so on) than the
	production server, which is where this Ansible deployment program comes in.
	All of this showed me how a proper deployment setup works, and the reasons
	why it's necessary --- a staging server running on completely separate
	systems (going as far as firewalling off the production systems to be
	absolutely sure) allows you to test things safely and not have to worry
	about serious bugs affecting customers.

\paragraphnl{December}
	This month was very quiet in terms of support calls and tickets (my
	colleagues say this was expected). I did a certain amount of time away from
	the office working from home as well. The low volume of support work meant
	that I had to find other work to do, be it learning about different sysadmin
	tools and technologies or development work. Working from home made me learn
	how to manage my time effectively and not get (or be) distracted.

\paragraphnl{January}
	Notable this month was me travelling to Leeds with Bytemark's Network
	Engineer, to help him install a new switch as part of expanding Bytemark's
	`core network'. This was a good opportunity to learn more about how the
	Internet works at a very abstract level, where all the ISPs peer with (or
	`connect to') each other. This essentially works by different ISPs linking
	together and then advertising, through BGP with its route preference system,
	where to find each others IP ranges. This propagates through the routers of
	all the world's ISPs, which is why you can access any site on the Internet
	from anywhere else. This area is very interesting to me as it actually
	involves the `people who make the internet work' and I would like to learn
	more about it later on.

\paragraphnl{February}
	I had another review this month, which overall was very positive. The main
	take away from it I had was that I need to improve how I conduct my phone
	calls. My line manager had no issue with what I actually said, rather the
	pauses inbetween when I was saying anything. I tended to `um' and `er' a lot
	when thinking about the correct answer to a customer's query. Reducing that
	will help customers feel more at ease when sorting out their problems and
	help reassure them that I know what I am doing (even when I don't!). I will
	try to reduce the amount of `thinking noise' I make over the phone in the
	future.

\paragraphnl{March}
	As part of my dev work this month it became necessary to test on an actual
	database server, as opposed to my own local copy. There are a number of
	reasons for this, notably to test my own database connection code works
	beyond `localhost'. This means connecting to separate databases for the
	development and production environments, as they're strictly kept separate.
	This presented an issue for one of the databases I was pulling data from, as
	no development version of it existed. Rather than just `invent' a database
	for the process, I decided it would be a good idea to look into the program
	that generated and used said database. Unfortunately this program is about
	as old as Bytemark itself, uses a framework that was old 5 years ago and a
	fair amount of technical debt. I spoke to several colleagues about the
	possibility of improving it and generally bringing its codebase into this
	decade, but while they agreed that it should be done, they also told me that
	I shouldn't waste my time trying. Despite this and me not having any
	significant experience in Ruby, the language it was written in, I spent
	several hours playing around with it and seeing how far I could get.
	Eventually I had to concede that it was above my knowledge level to fix and
	was forced to return to my IPs project. The experience made me realise that
	I shouldn't distract myself with what would've been a very large project and
	I should focus on the task assigned to me. Despite this, I think I'll return
	to this later to see if I can do any better. At this point it was decided to
	just place a copy of the production database onto the development system,
	still keeping data completely separate but meaning that it's accessible to
	development systems. Sometimes it's better to just get stuff done than to do
	it properly, if you know you're going to come back to it later.

	At the tail end of this month Bytemark had several people join the support
	team. This meant that the team had to make sure that everything was in place
	to train them efficently and give them access to the systems that they would
	need access to. It was interesting to me, after being at the company for 9
	months to be watching new people being taught how Bytemark works and the
	systems that they'd be using, and even doing a bit of training myself.

\paragraphnl{April}
	At this point my IPs project was nearing completion. However, it was being
	very slow to complete its conversion from one DB layout to the new one.
	Profiling my code revealed that it wasn't the code itself that was slow,
	rather its communication with the new database layer, which was done via an
	XML-RPC API that the new software provides. After some discussions with the
	senior developer, it was agreed that I should look into directly connecting
	to the backend database of the new program which would in theory be faster
	than the XML-RPC interface. After an inspection of the database's schema and
	of what the backend of the XML-RPC interface actually did to put the data
	into the database. After converting my program to do this, there was indeed
	a significant speed increase, albeit with profiling showing that it was
	again bottlenecked by the database connection as expected. This showed me
	that when programming it is important to identify where code is slowing down
	the program, understand why, and consider all possible solutions to improve
	it.

	During this rewrite process, I showed some particularly horrible code I'd
	written to the senior developer. The code itself was building SQL queries,
	but in such a way that the external data wasn't being sanitized at all. This
	is a big no-no with SQL, as it leaves you open to SQL injections. Despite
	this code never being run in the first place, my colleague was very
	displeased with this and insisted that I remove it and rewrite it properly
	using `prepared statements' before going any further. This taught me that
	code quality is important, as even though in this case there was almost no
	real danger (the data was going from one database to another) and that I
	should never cut corners when programming, especially when security is
	involved. This is important to the company as strong rules regarding coding
	helps reduce the number of potential vulnerabilities in end-user facing
	code.

	This month Bytemark had an all-company meeting that was special enough that
	it was spread over 2 days and had been given the name `Bytefest'. The first
	day had several talks, one by an outside speaker on the importance of
	staying relevant in such a fast paced industry. The computer industry has
	been very fast paced as long as it's been around and if you do not go along
	with these changes you can get left behind --- for an individual a lack of
	knowledge would make it difficult to get jobs and for a company a lack of
	new customers and revenue. There was also an update meeting by the
	Bytemark's directors, the main takeaway from which was some exciting
	prospects for Bytemark's future, with lots of growth planned for the next
	few years. They explained that many companies stay small, around Bytemark's
	size, while others are much bigger --- there aren't many in between this
	size. Growing the company by the amount planned would mean a lot of changes
	to the company internally which would be an interesting challenge to go
	through. The second day was primarily talks by colleagues, explaining what
	their main area of interest was and how it worked. This was useful as it
	gave other colleagues a greater understanding of how that area worked and
	will help them in their own contexts or projects. Through all of this, I was
	on frontline support, so I had to manage my time between watching the ticket
	queue for anything important (it had already been agreed that the queues
	didn't have to be watched quite as closely as usual for the couple of days)
	and between being part of the meeting and going to the talks.


%\paragraphnl{May}
%\paragraphnl{June}
%\paragraphnl{July}


\end{document}
